% 中文
\usepackage{ctex}
% 浮动体控制
\usepackage{float}
% 子图
\usepackage{subfigure}
% 数学相关
\usepackage{amsthm,amsmath,amssymb,mathrsfs}
% 页面
\usepackage[left=3cm,right=2.5cm,top=2.5cm,bottom=2.5cm]{geometry}
% 图片宏包
\usepackage{graphicx}
% 图片文件夹路径
\graphicspath{{../Output/}}
% 文献引用风格
\usepackage{natbib}
\usepackage[number]{gbt7714}
% 解决字体警告
\usepackage{anyfontsize}
% 积分算子
\newcommand{\dt}[1]{\frac{\mathrm{d}#1}{\mathrm{d}t}}
% 概率
\renewcommand{\P}[2]{P^{#1}_{#2}}
% 人群
\newcommand{\T}[2]{T^{#1}_{#2}}
% 感染机制
\newcommand{\TP}[3]{\T{#1}{#2}&=\P{#1}{#2}#3}
% 概率文本
\newcommand{\PText}[2]{$#1$群体转变为$#2$群体的概率}
% 摘要强调
\newcommand{\absemph}[1]{\textbf{#1}}
% 图片宽度
\def\imagewidth{0.95\textwidth}
% \def\imagewidth{0.45\textwidth}
\def\smallimagewidth{0.45\textwidth}
% 显示表格
\newcommand{\showtable}[1]{
    \begin{table}[H]
        \centering
        \caption{#1模型拟合参数}
        \input{Table/#1.tex}
    \end{table}
}
% 显示隔离后表格
\newcommand{\showtables}[1]{
    \begin{table}[H]
        \centering
        \caption{#1模型隔离前后拟合参数}
        \input{Table/#1隔离.tex}
    \end{table}
}
% 显示图片
\newcommand{\showfigure}[1]{
    \begin{center}
        \showtable{#1}
        \begin{figure}[H]
            \centering
            \includegraphics[width=\imagewidth]{#1.png}
            \caption{#1模型拟合图像}
        \end{figure}
    \end{center}
}
% 显示隔离后图片
\newcommand{\showfigures}[1]{
    \begin{minipage}{\imagewidth}
        \centering
        \showtables{#1}
        \begin{figure}[H]
            \centering
            \subfigure{
                \includegraphics[width=\smallimagewidth]{#1_隔离前.png}
            }
            \subfigure{
                \includegraphics[width=\smallimagewidth]{#1_隔离后.png}
            }
            \caption{#1模型隔离前后拟合图像}
        \end{figure}
    \end{minipage}
}
% SIR模型
\def\SIR{
    \begin{table}[H]
        \centering
        \caption{SIR模型参数表}
        \begin{tabular}{ll}
            \hline
            符号     & 含义         \\
            \hline
            $\alpha$ & \PText{S}{I} \\
            $\beta$  & \PText{I}{R} \\
            \hline
        \end{tabular}
    \end{table}
    \def\SI{IS\alpha}
    \def\IR{I\beta}
    \begin{align}
        \dt{S} & = -\SI      \\
        \dt{I} & = \SI - \IR \\
        \dt{R} & = \IR
    \end{align}
}
\def\SEIR{
    \begin{table}[H]
        \centering
        \caption{SEIR模型参数表}
        \begin{tabular}{ll}
            \hline
            符号       & 含义         \\
            \hline
            $\alpha$   & \PText{S}{E} \\
            $\gamma$   & \PText{E}{I} \\
            $\epsilon$ & \PText{E}{R} \\
            $\beta$    & \PText{I}{R} \\
            \hline
        \end{tabular}
    \end{table}
    \def\SE{(I+E)S\alpha}
    \def\EI{E\gamma}
    \def\ER{E\epsilon}
    \def\IR{I\beta}
    \begin{align}
        \dt{S} & = -\SE            \\
        \dt{E} & = \SE - \EI - \ER \\
        \dt{I} & = \EI - \IR       \\
        \dt{R} & = \IR +\ER
    \end{align}
}
\def\SEIRD{
    \begin{table}[H]
        \centering
        \caption{SEIRD模型参数表}
        \begin{tabular}{ll}
            \hline
            符号       & 含义         \\
            \hline
            $\alpha$   & \PText{S}{E} \\
            $\gamma$   & \PText{E}{I} \\
            $\epsilon$ & \PText{E}{R} \\
            $\beta$    & \PText{I}{R} \\
            $\delta$   & \PText{I}{D} \\
            \hline
        \end{tabular}
    \end{table}
    \def\SE{(I+E)S\alpha}
    \def\EI{E\gamma}
    \def\ER{E\epsilon}
    \def\IR{I\beta}
    \def\ID{I\delta}
    \begin{align}
        \dt{S} & = -\SE            \\
        \dt{E} & = \SE - \EI - \ER \\
        \dt{I} & = \EI - \IR - \ID \\
        \dt{R} & = \IR + \ER       \\
        \dt{D} & = \ID
    \end{align}
}
\def\SEIRS{
    \begin{table}[H]
        \centering
        \caption{SEIRS模型参数表}
        \begin{tabular}{ll}
            \hline
            符号       & 含义         \\
            \hline
            $\alpha$   & \PText{S}{E} \\
            $\gamma$   & \PText{E}{I} \\
            $\epsilon$ & \PText{E}{R} \\
            $\beta$    & \PText{I}{R} \\
            $\delta$   & \PText{I}{D} \\
            $\theta$   & \PText{R}{I} \\
            \hline
        \end{tabular}
    \end{table}
    \def\SE{(I+E)S\alpha}
    \def\EI{E\gamma}
    \def\ER{E\epsilon}
    \def\IR{I\beta}
    \def\ID{I\delta}
    \def\RI{R\theta}
    \begin{align}
        \dt{S} & = -\SE                  \\
        \dt{E} & = \SE - \EI - \ER       \\
        \dt{I} & = \RI + \EI - \IR - \ID \\
        \dt{R} & = \IR - \RI + \ER       \\
        \dt{D} & = \ID                   \\
    \end{align}
}