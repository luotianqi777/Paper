% 模型结果
\section{总结}
\section{文献综述}
% \citeauthor{对流行病数学理论的贡献}在\citeyear{对流行病数学理论的贡献}年针对传染病问题提出了$Compartmental$模型\cite{对流行病数学理论的贡献},将人群划分为多个部分,并假设同一部分中的每个人都具有相同的特征从而建立积分方程推导传染病的传播过程;
% \citeauthor{Kermack-McKendrick确定性流行病模型的推广}于\citeyear{Kermack-McKendrick确定性流行病模型的推广}年对$Compartmental$模型做出了推广,将人群为易感染者$S$,感染者$I$,康复者$R$,提出了经典的$SIR$模型\cite{Kermack-McKendrick确定性流行病模型的推广},后来的$SEIR$,$SEIS$,$MSEIR$等众多模型都是以$SIR$模型为基础衍生出来的;
% 王小莉等人于\citeyear{应用SEIR模型预测2009年甲型H1N1流感流行趋势}年根据不同因素下(年龄、地区、生活习惯等)的数据建立$SEIR$模型,探究预测甲型H1N1流感在不同因素下的传播情况\cite{应用SEIR模型预测2009年甲型H1N1流感流行趋势};
% 杨旭颖等人于\citeyear{基于SEIR的社交网络信息传播模型}年将$SEIR$模型用于模拟社交网络的信息传播机制\cite{基于SEIR的社交网络信息传播模型},目的是预测社交网络上的信息传播趋势;
% 王晓红于\citeyear{一类具有潜伏期的SEIR手足口病模型的研究}年将$SEIR$模型用于预测手足口病的疫情趋势\cite{一类具有潜伏期的SEIR手足口病模型的研究},并考虑了康复者复发的情况(即后来的$SEIRS$模型),得到了足口病的基本再生数并对模型进行了动力学分析:平衡点的存在性和稳定性;
% 崔金栋等人于\citeyear{基于改良SEIR模型的微博话题式信息传播研究}年以微博信息传播中的SEIR模型为出发点,综合考虑微博网络中话题式信息的衍生特性,构建改良式的微博话题式信息传播H-SEIR模型\cite{基于改良SEIR模型的微博话题式信息传播研究},并运用MATLAB进行模拟仿真,对微博中话题式信息传播影响因素和对应的控制策略进行研究;
% 赵世磊等人通过$SUQC$模型\cite{通过流行病学建模表征传播和确定COVID-19的控制策略}(在$SEIR$模型基础上加入潜伏期感染和隔离条件)基于官方每日发布的数据不断调整隔离参数及传染率,最终得出疫情在不同阶段的隔离率,确诊率及病毒生殖率(即传播率)并对疫情最终影响人数做出预测,其模型较为单一,结果可能不够准确,但将参数分段考虑的方式值得借鉴;
% 李静华运用五种不同的方法\cite{估计2019年新型冠状病毒的流行性:数学建模研究}(分别是指数增长,最大似然法,顺序贝叶斯方法,时间序列分析,$SEIR$模型)对病毒在2月8日前后及整个疫情期间的传播率做出了估计,结论较为可靠,但并没有进一步预测未来的疫情趋势;
% 周慧娟等人通过$SEIQ$模型\cite{中国COVID-19爆发的流行动力学模型和控制}(在$SEIR$模型基础上加入隔离人群$Q$)通过$Montecarlo$方法模拟了两个独立的泊松过程(分别是每日暴露病例和个体潜伏时间)讨论在不同环境的不同隔离措施下病毒的感染率;
% 金华潘等人通过对$SEIR$模型做出一系列改动\cite{2019年冠状病毒疾病控制策略的有效性:SEIR动态建模研究}(其中包括超级传染,治疗等)计算了2月16日之前的病毒基础生殖率,但该模型并未考虑隔离情况,因此预测可能有较大偏差,但其对治疗人群的引入值得借鉴;
% \citeauthor{估计2020年公主邮轮船上2019年新型冠状病毒的无症状比率}使用哈密顿蒙特卡洛($HMC$)的贝叶斯框架\cite{估计2020年公主邮轮船上2019年新型冠状病毒的无症状比率}预测封闭的小范围环境(游轮环境)的未感染人数,可以用来粗略估计封闭的大范围环境(封闭下的城市)的未感染人数;
% 战军湛等人基于$SEIR$模型借助人口迁移学通过城市间人口流动数据\cite{结合人类迁移数据的2019年冠状病毒疾病在中国传播的建模与预测}预测疫情在城市间的传播,但没有考虑隔离的情况并轻视了疫情期间人口流动严重停滞的情况,故其预测结果偏差较大,不可用于预测结果参考,但其以城市为基本单元的思考方式值得借鉴;
% 彭鹏等人基于广义动力学模型\cite{基于动力学模型的中国COVID-19流行病学分析}引入七个状态(在原有的四个状态基础上新加入隔离,死亡,免疫)结合官方发布数据,预测疫情进展并对疫情持续时间做出估计,可用于结果评测;
% 张良禄等人通过基础$SEIR$建模\cite{SEIR建模分析与新型冠状病毒的斗争何时在武汉结束},将1月22日到2月12日间的数据以2月7日为界限分别建立了模型,得出一系列基础参数并给出了预测结果,但其对疫情过于乐观,并未考虑隔离及潜伏期可传染,不适用于长期预测;
% \citeauthor{协调基本生殖数量及其不确定性的早期暴发估计:新型冠状病毒(SARS-CoV-2)暴发的框架和应用}认为单纯的将病毒生殖数看作一个常量已不适用于COVID-19,故将病毒基本生殖数划分为三个关键量\cite{协调基本生殖数量及其不确定性的早期暴发估计:新型冠状病毒(SARS-CoV-2)暴发的框架和应用}(指数增长率,平均生成间隔,生成间隔离散度)并结合官方数据来估算病毒的传染率,其思想及结果值得借鉴,或许将一些参数(如:传染率,治愈率)设计为一个与时间有关的分段函数(以隔离措施及医学进展的时间点为断点)比较合适。
