\begin{appendix}
    \section{代码}
    \subsection{基础类}
    \begin{itemize}
        \item BaseClass:基础类,
              定义一系列接口,
              方便管理其他模型。
        \item DataManager:数据管理类,
              继承自BaseClass,
              用于保存及读取csv数据。
        \item Drawer:绘画类,
              继承自BaseClass,
              用于绘制图像。
        \item TexTabelBulier:表格类,
              继承自BaseClass,
              用于将参数数据保存为LaTeX表格。
    \end{itemize}
    \lstinputlisting{../Python/baseClass.py}
    \subsection{数据获取}
    \begin{itemize}
        \item DataCrawler:数据爬虫类,
              继承自DataManager,
              用于获取截止当日的疫情数据,
              并提供读取功能。
    \end{itemize}
    \lstinputlisting{../Python/dataCrawler.py}
    \subsection{数据分析}
    \begin{itemize}
        \item DataAnalysis:数据分析类,
              继承自Drawer,
              可以方便的分析处理数据。
    \end{itemize}
    \lstinputlisting{../Python/dataAnalysis.py}
    \subsection{模型及拟合}
    \begin{itemize}
        \item Model:模型基础类,
              继承自Drawer,
              定义一系列接口,
              可以方便的创建新模型,
              提供十分快捷的拟合功能。
        \item SIR:SIR类,继承自Model,实现SIR模型。
        \item SEIR:SEIR类,继承自Model,实现SEIR模型。
        \item SEIRD:SEIRD类,继承自Model,实现SEIRD模型。
        \item SEIRS:SEIRS类,继承自Model,实现SEIRS模型。
    \end{itemize}
    \lstinputlisting{../Python/model.py}
    \subsection{主函数}
    \par 用于启动整个程序。
    \lstinputlisting{../Python/main.py}
\end{appendix}