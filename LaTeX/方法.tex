\section{研究方法}
\subsection{研究框架}
\subsection{模型讨论}
\par 经典SIR模型的表述方式简洁直观,但在扩建模型时会产生一些问题:
由表\ref{table:SIR模型符号表}可预见,随着划分人群的增多,人群间感染机制也随之复杂,不同人群之间转变率的符号也会增多或者改变含义,这会使读者对符号的理解产生负面的路径依赖影响,对于模型的扩建是十分不利的。
\par 本文将采用另一种较为抽象的模型表述方式,以此来减轻读者的思维负担。
\begin{table}[H]
    \centering
    \caption{另一种模型符号表示}
    \begin{tabular}{ll}
        \hline
        符号       & 含义                              \\
        \hline
        S          & 易感者                            \\
        I          & 感染者                            \\
        R          & 康复者                            \\
        $\P{a}{b}$ & a群体变为b群体的概率              \\
        $\T{a}{b}$ & 单位时间$t$内a群体变为b群体的人数 \\
        \hline
    \end{tabular}
\end{table}
\par 将a到b群体的转换概率用$\P{a}{b}$表示,单位时间内a到b群体转变的人数为$\T{a}{b}$。
\par 感染机制如下:
\begin{align}
    S(t)\xrightarrow{\P{S}{I}}I(t) \Rightarrow \TP{S}{I}{SI} \\
    I(t)\xrightarrow{\P{I}{R}}R(t) \Rightarrow \TP{I}{R}{I}
\end{align}
\par 可将其简化为:
\begin{align}
    \TP{S}{I}{SI} \\
    \TP{I}{R}{I}
\end{align}
\par 将所有人群放到一个集合$\mathbb{A}$中,则模型的积分方程为:
\begin{equation}
    \dt{a} = \sum\left(\T{b}{a}-\T{a}{b}\right)
\end{equation}
\par 其中$a\in\mathbb{A}$,$b\in\mathbb{A}$且$b\not=a$,不在感染机制中的$\T{a}{b}$为$0$。
\par 该积分式对本文中提到的所有模型都适用,
这意味着我们将不必再关心模型的积分式,
只需关注模型新引入的群体及其引起的感染机制$\T{a}{b}$变化即可。
\par 这种描述方式有一个缺点,只能用于描述总量恒定且参数固定的模型。
\par 现在,可以仅通过感染机制来方便清晰的描述$SIR$模型:
\begin{align}
    \TP{S}{I}{SI} \\
    \TP{I}{R}{I}
\end{align}
\subsection{模型推论}
考虑到易感人群接触到感染者后不会立即患病,
而是经过一段时间潜伏期,
即携带病毒还未患病,
将该类人群定义为携带者人群$E$,
即为$SEIR$模型。
在COVID-19中这类人群一般会通过检测试剂等方式被诊断为疑似病例:
\begin{align}
    \TP{S}{E}{S(I+E)} \\
    \TP{E}{I}{E}      \\
    \TP{I}{R}{I}
\end{align}
在$SEIR$的基础上加入死亡人群$D$,
即为$SEIRD$模型:
\begin{align}
    \TP{S}{E}{S(I+E)} \\
    \TP{E}{I}{E}      \\
    \TP{I}{R}{I}      \\
    \TP{I}{D}{I}
\end{align}
考虑到治愈者有复发的可能,
康复者有一定比例重新转变为感染者,
加入新的传播机制$\T{R}{I}$,
即为$SEIRS$模型:
\begin{align}
    \TP{S}{E}{S(I+E)} \\
    \TP{E}{I}{E}      \\
    \TP{I}{R}{I}      \\
    \TP{I}{D}{I}      \\
    \TP{R}{I}{R}
\end{align}
\subsection{数据拟合结果}
\subsubsection{SIR}
\showfigure{SIR}
\showfigures{SIR}
\subsubsection{SEIR}
\showfigure{SEIR}
\showfigures{SEIR}
\subsubsection{SEIRD}
\showfigure{SEIRD}
\showfigures{SEIRD}
\subsubsection{SEIRS}
\showfigure{SEIRS}
\showfigures{SEIRS}