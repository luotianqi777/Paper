\section{结论}
\subsection{总结}
\par 本文在$SIR$模型的基础上根据病毒特点扩建多个数学模型,
基于官方发布的$COVID-19$疫情数据,
对不同防控阶段的疫情传播趋势进行总体及分段拟合,
通过对拟合结果进行对比选出最适合该疫情的模型$SEIRD$
(详见图\ref{figure:SEIRD模型拟合图像}、
图\ref{figure:SEIRD模型隔离前后拟合图像}),
并根据最优模型的最优估计参数
(详见表\ref{table:SEIRD模型拟合参数}、
表\ref{table:SEIRD模型隔离前后拟合参数})
计算出整个疫情期间的基本再生数$R_0=1.754$,
计算结果与官方发布数据基本相符,
以及早期未采取隔离措施时的$R_0=5.937$,
采取隔离措施后的$R_0=1.081$。
\subsection{创新点}
\par 将隔离前后的数据分开拟合,
用于计算隔离前后的病毒基本生殖数。
\subsection{不足及展望}
\par 模型所采用参数均为固定值,
实际情况有可能为随时间动态变化的序列。
本文结论仅适用于优化参数为固定值的模型。
\par 由于城市间疫情情况差距较大,
对疫情严重区域进行拟合更能反映出疫情的传播趋势,
但本文使用全国范围数据,
结论可能与真实情况有一定偏差。
\par 因为疫情已近尾声,
本文并未对疫情做出预测。
但可以使用其他国家地区的疫情数据对疫情做出预测。