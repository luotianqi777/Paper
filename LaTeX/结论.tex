\section{结论}
\subsection{总结}
\par 本文在$SEIR$模型的基础上根据病毒特点创建多个衍生模型,
基于官方发布的COVID-19疫情数据,
对不同防控阶段的疫情传播趋势进行总体及分段拟合,
通过对拟合结果进行对比发现加入死亡人群$D$的$SEIR$模型与真实数据拟合度最好
(详见图\ref{figure:SEIRD模型拟合图像}、
图\ref{figure:SEIRD模型隔离前后拟合图像}),
根据最优模型的最优估计参数
(详见表\ref{table:SEIRD模型拟合参数}、
表\ref{table:SEIRD模型隔离前后拟合参数})
计算出整个疫情期间的$R_t=\rot$,
计算结果与官方发布数据基本相符,
以及早期未采取隔离措施时的$R_0=\ro$,
采取隔离措施后的$R_t=\rt$。
\subsection{不足及展望}
\par 模型所采用参数均为固定值,
实际情况有可能为随时间动态变化的序列,
但未能找到$SEIR$及其衍生模型拟合动态参数的方法,
故无法对其进行细致分析。
因此本文结论只能描述疫情宏观的传播情况。
\par 由于城市间疫情情况差距较大,
对疫情严重区域进行拟合更能反映出疫情的传播趋势,
但由于时间较为紧迫且未能找到方便爬取的省份数据源,
本文决定采用全国范围数据,
结论可能与真实情况有一定偏差。
\par 实际情况中存在境外输入病例,
但其输入时间及人数均不稳定,
甚至无法找到其规律,
本人能力不足以给出一个可信度高的模拟境外输入病例的模型,
故本文结论可能不适用于输入病例频繁的地区。