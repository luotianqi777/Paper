\section{模型研究}
\subsection{模型解释}
\par 通过微分方程来描述美观简练,
但在扩建模型时会有一些麻烦:
由表\ref{table:SIR模型符号表}可预见,
随着划分人群的增多,
人群间感染机制也随之复杂,
不同人群之间转变率的符号也会增多或者改变含义,
这会使读者对符号的理解产生负面的路径依赖影响,
对于模型的扩建是十分不利的。
\par $SIR$模型的本质为人群间的转移,
只要理解人群间的转移方程便可了解模型的结构。
而这类模型都是在总人数固定的前提下进行的,
所以通过状态转移方程即可得出微分方程,
传染病问题中,
状态转移等价于感染机制,
为了更为清晰的描述人群间的感染机制,
本文将引入新的描述方式。
\begin{table}[H]
    \centering
    \caption{另一种模型符号表示}
    \begin{tabular}{ll}
        \hline
        符号       & 含义                              \\
        \hline
        $\P{a}{b}$ & a群体变为b群体的概率              \\
        $\T{a}{b}$ & 单位时间$t$内a群体变为b群体的人数 \\
        \hline
    \end{tabular}
\end{table}
\par 将a到b群体的转换概率用$\P{a}{b}$表示,单位时间内a到b群体转变的人数为$\T{a}{b}$。
\par 感染机制如下:
\begin{align}
    S(t)\xrightarrow{\P{S}{I}}I(t) \Rightarrow \TP{S}{I}{SI} \\
    I(t)\xrightarrow{\P{I}{R}}R(t) \Rightarrow \TP{I}{R}{I}
\end{align}
\par 可将其简化为:
\begin{align}
    \TP{S}{I}{SI} \\
    \TP{I}{R}{I}
\end{align}
\par 易感者转变为感染者,感染者转变为治愈者(包含死亡者)。
\par 将所有人群放到一个集合$\mathbb{A}$中,则模型的积分方程为:
\begin{equation}
    \dt{a} = \sum\left(\T{b}{a}-\T{a}{b}\right)
\end{equation}
\par 其中$a\in\mathbb{A}$,$b\in\mathbb{A}$且$b\not=a$,不在感染机制中的$\T{a}{b}$为$0$。
\par 该积分式对本文中提到的所有模型都适用,
读者通过了解新模型的感染机制$\T{a}{b}$即可了解该模型的结构,
进而宏观的理解整个模型的运作方式。
本文会在介绍模型时首先列出感染机制,
随后给出参数表及详细积分式供读者参考。
\subsection{模型推论}
\par 本文共引入$5$种人群,在此声明。
\begin{table}[H]
    \centering
    \caption{人群}
    \begin{tabular}{ll}
        \hline
        符号 & 含义   \\
        \hline
        $S$  & 易感者 \\
        $I$  & 感染者 \\
        $R$  & 康复者 \\
        $E$  & 携带者 \\
        $D$  & 病逝者 \\
        \hline
    \end{tabular}
\end{table}
\subsubsection{$SEIR$模型}
\par 考虑到易感人群接触到感染者后不会立即患病,
而是经过一段时间潜伏期,
即携带病毒还未患病,
将该类人群定义为携带者人群$E$,
该人群有可能转变为治愈者$R$或感染者$I$,
即为$SEIR$模型。
在COVID-19中这类人群一般会通过检测试剂等方式被诊断为疑似病例。
\paragraph{感染机制}
\begin{align}
    \TP{S}{E}{S(I+E)} \\
    \TP{E}{R}{E}      \\
    \TP{E}{I}{E}      \\
    \TP{I}{R}{I}
\end{align}
\par 易感者会转变为携带者,
由于接触携带者也会感染,
故此时接触者为$I+E$,
携带者会转变为治愈者或感染者。
\paragraph{详细积分式}
\SEIR
\subsubsection{$SEIRD$模型}
\par 在$SEIR$的基础上加入死亡人群$D$,
即为$SEIRD$模型。
\paragraph{感染机制}
\begin{align}
    \TP{S}{E}{S(I+E)} \\
    \TP{E}{R}{E}      \\
    \TP{E}{I}{E}      \\
    \TP{I}{R}{I}      \\
    \TP{I}{D}{I}
\end{align}
\par 感染者会转变为治愈者或病逝者。
\paragraph{详细积分式}
\SEIRD
\subsubsection{$SEIRS$模型}
\par 考虑到治愈者有复发的可能,
康复者有一定比例重新转变为感染者,
在$SEIRD$模型中加入新的传播机制$\T{R}{I}$,
即为$SEIRS$模型。
\paragraph{感染机制}
\begin{align}
    \TP{S}{E}{S(I+E)} \\
    \TP{E}{R}{E}      \\
    \TP{E}{I}{E}      \\
    \TP{I}{R}{I}      \\
    \TP{I}{D}{I}      \\
    \TP{R}{I}{R}
\end{align}
\par 治愈者有可能变为感染者,
之所以不是携带者是因为治愈者已有抗体,
由携带到感染的几率小于易感者,
故感染病毒后不能简单的归入携带者,
所以将其从携带到感染归为一个过程。
\paragraph{详细积分式}
\SEIRS
\subsection{对隔离的处理}
\par 本文并未尝试引入隔离群体的$SEIHR$模型,
因为在该数据中隔离群体是携带者、感染者的混合群体,
且隔离群体有可能治愈或死亡,
而这其中的比例无法通过现有数据得知,
故不能通过拟合来确定其参数。
\par 对于隔离的处理,
本文将数据分为隔离前与隔离后两部分,
对这两个时间段的数据分别进行拟合,
以此来对比隔离前后对疫情传播的影响。