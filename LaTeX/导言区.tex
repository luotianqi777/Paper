% 中文
\usepackage{ctex}
% 浮动体控制
\usepackage{float}
% 超链接
\usepackage[
    colorlinks,
    linkcolor=black,
    anchorcolor=black,
    citecolor=black
]{hyperref}
% 间距控制
\usepackage{setspace}
% 子图
\usepackage{subfigure}
% 数学相关
\usepackage{amsthm,amsmath,amssymb,mathrsfs}
% 图表公式章节编号
\renewcommand{\thetable}{\thesection-\arabic{table}}
\renewcommand{\thefigure}{\thesection-\arabic{figure}}
\renewcommand{\theequation}{\thesection.\arabic{equation}}
\makeatletter
\@addtoreset{table}{section}
\@addtoreset{figure}{section}
\@addtoreset{equation}{section}
\makeatother
% 页面
\usepackage[left=2cm,right=2cm,top=2.5cm,bottom=2cm]{geometry}
% 解决字体警告
\usepackage{anyfontsize}
% 设置英文字体
\setmainfont{Times New Roman}
% 图片宏包
\usepackage{graphicx}
% 图片文件夹路径
\graphicspath{{../Output/}}
% 文献引用风格
\usepackage{natbib}
% 规定论文格式
\usepackage[number]{gbt7714}
% 代码环境
\usepackage{listings}
% 颜色
\usepackage{xcolor}
% 字体
\usepackage{fontspec}
% 代码格式设置
\lstset{
    % 语言
    language=python,
    % 自动换行
    breaklines=true,
    % 行号
    numbers=left,
    % 字符串显示空格
    showstringspaces=false,
    % 基础字体设置
    basicstyle=\small\fontspec{Monaco},
    % 字符串风格
    stringstyle=\color{purple},
    % 注释风格
    commentstyle=\color{gray},
    % 关键字风格
    keywordstyle=\color{blue},
    % 左侧margin
    xleftmargin = 25pt,
    % 添加frame
    frame = tb,
    % 设置frame左侧margin
    framexleftmargin = 20pt,
}
% 流程图
\usepackage{tikz}
\usetikzlibrary{shapes.geometric,arrows}
\tikzstyle{startstop} = [rectangle,rounded corners, minimum width=1cm,minimum height=1cm,text centered, draw=black,fill=red!30]
\tikzstyle{io} = [trapezium, trapezium left angle = 70,trapezium right angle=110,minimum width=1cm,minimum height=1cm,text centered,draw=black,fill=blue!30]
\tikzstyle{process} = [rectangle,minimum width=1cm,minimum height=1cm,text centered,text width =3cm,draw=black,fill=orange!30]
\tikzstyle{decision} = [diamond,minimum width=1cm,minimum height=1cm,text centered,draw=black,fill=green!30]
\tikzstyle{arrow} = [thick,->,>=stealth]
% 积分算子
\newcommand{\dt}[1]{\frac{\mathrm{d}#1}{\mathrm{d}t}}
% 概率
\renewcommand{\P}[2]{P^{#1}_{#2}}
% 人群
\newcommand{\T}[2]{T^{#1}_{#2}}
% 感染机制
\newcommand{\TP}[3]{\T{#1}{#2}&=\P{#1}{#2}#3}
% 概率文本
\newcommand{\PText}[2]{$#1$群体转变为$#2$群体的概率}
% 摘要强调
\newcommand{\absemph}[1]{\textbf{#1}}
% 英文摘要
\newenvironment{enabstract}{
    \par\small
    \noindent\mbox{}\hfill{\bfseries \zihao{-3} ABSTRACT}\hfill\mbox{}\par
    \vskip 2.5ex}{\par\vskip 2.5ex}
% 中文摘要
\newenvironment{cnabstract}{
    \par\small
    \noindent\mbox{}\hfill{\bfseries \heiti\zihao{-3} 摘  要}\hfill\mbox{}\par
    \vskip 2.5ex}{\par\vskip 2.5ex}
% 图片宽度
\def\imagewidth{0.65\textwidth}
\def\smallimagewidth{0.45\textwidth}
% 显示表格
\newcommand{\showtable}[1]{
    \begin{table}[H]
        \centering
        \caption{#1模型拟合参数\label{table:#1模型拟合参数}}
        \input{Table/#1.tex}
    \end{table}
}
% 显示隔离后表格
\newcommand{\showtables}[1]{
    \begin{table}[H]
        \centering
        \caption{#1模型隔离前后拟合参数\label{table:#1模型隔离前后拟合参数}}
        \input{Table/#1隔离.tex}
    \end{table}
}
% 显示图片
\newcommand{\showfigure}[1]{
    \begin{figure}[H]
        \centering
        \includegraphics[width=\imagewidth]{#1.png}
        \caption{#1模型拟合图像\label{figure:#1模型拟合图像}}
    \end{figure}
}
% 显示隔离后图片
\newcommand{\showfigures}[1]{
    \begin{figure}[H]
        \centering
        \subfigure{
            \includegraphics[width=\smallimagewidth]{#1_隔离前.png}
        }
        \subfigure{
            \includegraphics[width=\smallimagewidth]{#1_隔离后.png}
        }
        \caption{#1模型隔离前后拟合图像\label{figure:#1模型隔离前后拟合图像}}
    \end{figure}
}
% 导入预定义指令
% 该文件仅负责只适用于该论文的自定义命令
% 格式配置或通用指令放到 导言区.tex
% 积分算子
\newcommand{\dt}[1]{\frac{\mathrm{d}#1}{\mathrm{d}t}}
% 概率
\renewcommand{\P}[2]{P^{#1}_{#2}}
% 人群
\newcommand{\T}[2]{T^{#1}_{#2}}
% 感染机制
\newcommand{\TP}[3]{\T{#1}{#2}&=\P{#1}{#2}#3}
% 概率文本
\newcommand{\PText}[2]{$#1$群体转变为$#2$群体的概率}
% 全局有效再生数
\def\rot{1.588}
% 基本再生数
\def\ro{6.62}
% 有效再生数
\def\rt{1.573}
% 图片宽度
\def\imagewidth{0.65\textwidth}
\def\smallimagewidth{0.45\textwidth}
% 显示数据表格
\newcommand{\showdatatable}[1]{
    \input{Table/#1模型拟合数据.tex}
}
% 显示隔离前后数据表格
\newcommand{\showdatatables}[1]{
    \input{Table/#1隔离前模型拟合数据.tex}
    \input{Table/#1隔离后模型拟合数据.tex}
}
% 显示表格
\newcommand{\showtable}[1]{
    \begin{table}[H]
        \centering
        \caption{#1模型拟合参数\label{table:#1模型拟合参数}}
        \input{Table/#1.tex}
    \end{table}
}
% 显示隔离后表格
\newcommand{\showtables}[1]{
    \begin{table}[H]
        \centering
        \caption{#1模型隔离前后拟合参数\label{table:#1模型隔离前后拟合参数}}
        \input{Table/#1隔离.tex}
    \end{table}
}
% 显示图片
\newcommand{\showfigure}[1]{
    \begin{figure}[H]
        \centering
        \includegraphics[width=\imagewidth]{#1.png}
        \caption{#1模型拟合图像\label{figure:#1模型拟合图像}}
    \end{figure}
}
% 显示隔离后图片
\newcommand{\showfigures}[1]{
    \begin{figure}[H]
        \centering
        \subfigure[#1模型隔离前拟合图像]{
            \includegraphics[width=\smallimagewidth]{#1隔离前.png}
        }
        \subfigure[#1模型隔离后拟合图像]{
            \includegraphics[width=\smallimagewidth]{#1隔离后.png}
        }
        \caption{#1模型隔离前后拟合图像\label{figure:#1模型隔离前后拟合图像}}
    \end{figure}
}
% SIR模型
\def\SIR{
    \begin{table}[H]
        \centering
        \caption{SIR模型参数表}
        \begin{tabular}{ll}
            \hline
            符号     & 含义         \\
            \hline
            $\alpha$ & \PText{S}{I} \\
            $\beta$  & \PText{I}{R} \\
            \hline
        \end{tabular}
    \end{table}
    \def\SI{IS\alpha}
    \def\IR{I\beta}
    \begin{align}
        \dt{S} & = -\SI      \\
        \dt{I} & = \SI - \IR \\
        \dt{R} & = \IR
    \end{align}
}
\def\SEIR{
    \begin{table}[H]
        \centering
        \caption{SEIR模型参数表}
        \begin{tabular}{ll}
            \hline
            符号       & 含义         \\
            \hline
            $\alpha$   & \PText{S}{E} \\
            $\gamma$   & \PText{E}{I} \\
            $\epsilon$ & \PText{E}{R} \\
            $\beta$    & \PText{I}{R} \\
            \hline
        \end{tabular}
    \end{table}
    \def\SE{(I+E)S\alpha}
    \def\EI{E\gamma}
    \def\ER{E\epsilon}
    \def\IR{I\beta}
    \begin{align}
        \dt{S} & = -\SE            \\
        \dt{E} & = \SE - \EI - \ER \\
        \dt{I} & = \EI - \IR       \\
        \dt{R} & = \IR +\ER
    \end{align}
}
\def\SEIRD{
    \begin{table}[H]
        \centering
        \caption{SEIRD模型参数表}
        \begin{tabular}{ll}
            \hline
            符号       & 含义         \\
            \hline
            $\alpha$   & \PText{S}{E} \\
            $\gamma$   & \PText{E}{I} \\
            $\epsilon$ & \PText{E}{R} \\
            $\beta$    & \PText{I}{R} \\
            $\delta$   & \PText{I}{D} \\
            \hline
        \end{tabular}
    \end{table}
    \def\SE{(I+E)S\alpha}
    \def\EI{E\gamma}
    \def\ER{E\epsilon}
    \def\IR{I\beta}
    \def\ID{I\delta}
    \begin{align}
        \dt{S} & = -\SE            \\
        \dt{E} & = \SE - \EI - \ER \\
        \dt{I} & = \EI - \IR - \ID \\
        \dt{R} & = \IR + \ER       \\
        \dt{D} & = \ID
    \end{align}
}
\def\SEIRS{
    \begin{table}[H]
        \centering
        \caption{SEIRS模型参数表}
        \begin{tabular}{ll}
            \hline
            符号       & 含义         \\
            \hline
            $\alpha$   & \PText{S}{E} \\
            $\gamma$   & \PText{E}{I} \\
            $\epsilon$ & \PText{E}{R} \\
            $\beta$    & \PText{I}{R} \\
            $\delta$   & \PText{I}{D} \\
            $\theta$   & \PText{R}{I} \\
            \hline
        \end{tabular}
    \end{table}
    \def\SE{(I+E)S\alpha}
    \def\EI{E\gamma}
    \def\ER{E\epsilon}
    \def\IR{I\beta}
    \def\ID{I\delta}
    \def\RI{R\theta}
    \begin{align}
        \dt{S} & = -\SE                  \\
        \dt{E} & = \SE - \EI - \ER       \\
        \dt{I} & = \RI + \EI - \IR - \ID \\
        \dt{R} & = \IR - \RI + \ER       \\
        \dt{D} & = \ID
    \end{align}
}