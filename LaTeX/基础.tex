\section{研究基础}
\subsection{研究概况}
疫情发生以来,
众多国内外不同领域的专家学者对此高度关注,
从各个层面深度剖析,
运用各种方法预测疫情趋势。
其中以基于$SIR$模型的$SEIR$模型最为多见,
众多专家学者依据COVID-19的特点通过$SEIR$模型进行拟合分析,
得出诸多研究成果。
\subsection{文献综述}
\input{文献综述.tex}
\subsection{理论基础}
\par 在介绍$SEIR$模型前,
需要先说明$SIR$模型,
$SEIR$模型是由$SIR$模型发展而来的。
\par \citeauthor{对流行病数学理论的贡献}在\citeyear{对流行病数学理论的贡献}年研究黑死病时提出了仓室模型,模型中将人口分为三类:
\begin{itemize}
    \item 易感者(susceptibles),$S$人群
    \item 感染者(infectives),$I$人群
    \item 康复者(recovered),$R$人群
\end{itemize}
\par 将其称为$SIR$
\cite{对流行病数学理论的贡献}模型。
\citeauthor{Kermack-McKendrick确定性流行病模型的推广}在\citeyear{Kermack-McKendrick确定性流行病模型的推广}年对其进行了推广\cite{Kermack-McKendrick确定性流行病模型的推广},证明其广泛的适用性。
\par $SIR$模型的建立基于几个假设\cite{对流行病数学理论的贡献}:
\begin{itemize}
    \item 人口总数保持常量(包含预测死亡)
    \item 单位时间$t$传染人数与$S$和$I$人数成正比,即$S\to I = \alpha SI$
    \item 单位时间$t$康复人数与$I$成正比,即$I\to R = \beta I$
\end{itemize}
\begin{table}[H]
    \centering
    \caption{SIR模型符号表}
    \label{table:SIR模型符号表}
    \begin{tabular}{ll}
        \hline
        符号     & 含义         \\
        \hline
        S        & 易感者       \\
        I        & 感染者       \\
        R        & 康复者       \\
        $\alpha$ & \PText{S}{I} \\
        $\beta$  & \PText{I}{R} \\
        \hline
    \end{tabular}
\end{table}
\par 感染机制如下:
\begin{align}
    S(t) & \xrightarrow \alpha I(t) \\
    I(t) & \xrightarrow \beta R(t)
\end{align}
\par 可以用积分方程表示为
\begin{align}
    \dt{S} & = -\alpha SI          \label{math:SIR_S}  \\
    \dt{I} & = \alpha SI - \beta I  \label{math:SIR_I} \\
    \dt{R} & = \beta I\label{math:SIR_R}
\end{align}
\par 对式(\ref{math:SIR_S})的说明:
\par 易感者转变为感染者的人数为$\text{易感者}\times\text{接触率}\times\text{感染率}$,
其中接触率为$\frac{\text{感染者}}{\text{总人数}}$。
\par 总人数是一个常数,
规定$\alpha=\frac{\text{感染率}}{\text{总人数}}$,
得出$\dt{S}$:
\begin{align*}
    \dt{S} & = -\text{易感者}\times\text{接触率}\times\text{感染率} \\
           & = -S\cdot\frac{I}{\text{总人数}}\cdot\text{感染率}     \\
           & = -S\cdot I\cdot\frac{\text{感染率}}{\text{总人数}}    \\
           & = -S\cdot I \cdot \alpha
\end{align*}
\par 对式(\ref{math:SIR_R})的说明:
\begin{align*}
    \dt{R} & = \text{感染者}\times\text{治愈率} \\
           & =I\cdot \beta
\end{align*}
\par 对式(\ref{math:SIR_I})的说明:
\begin{align*}
    \dt{I} & = -\dt{S} - \dt{R}                     \\
           & = S\cdot I \cdot \alpha - I\cdot \beta
\end{align*}
\par $SEIR$模型是在$SIR$模型基础上加入病毒携带者$E$(Exposed)得来的,
详细模型将在\ref{sec:SEIR}进行说明。
\subsection{使用工具}
本文使用python实现爬虫动态获取所需数据、
使用scipy库进行高效率的积分求解及数据拟合、
使用pyecharts库绘制较为美观的图像。