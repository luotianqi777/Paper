\section{研究基础}
\subsection{研究概况}
疫情发生以来,
众多国内外不同领域的专家学者对此高度关注,
从各个层面深度剖析,
运用各种方法预测疫情趋势。
其中以流体动力学模型为基础的方法最为多见,
根据COVID-19的特点构建了许多模型,
得出众多研究成果。
\subsection{文献综述}
\input{文献综述.tex}
\subsection{理论基础}
\citeauthor{对流行病数学理论的贡献}在\citeyear{对流行病数学理论的贡献}年研究黑死病时提出了仓室模型,模型中将人口分为三类:
\begin{itemize}
    \item 易感者(susceptibles),$S$人群
    \item 感染者(infectives),$I$人群
    \item 康复者(recovered),$R$人群
\end{itemize}
\par 将其称为$SIR$
\cite{对流行病数学理论的贡献}模型。
\citeauthor{Kermack-McKendrick确定性流行病模型的推广}在\citeyear{Kermack-McKendrick确定性流行病模型的推广}年对其进行了推广\cite{Kermack-McKendrick确定性流行病模型的推广},证明其广泛的适用性。
\par $SIR$模型的建立基于几个假设\cite{对流行病数学理论的贡献}:
\begin{itemize}
    \item 人口总数保持常量(包含死亡人数)
    \item 单位时间$t$传染人数与$S$和$I$人数成正比,即$S\to I = \alpha SI$
    \item 单位时间$t$康复人数与$I$成正比,即$I\to R = \beta I$
\end{itemize}
\begin{table}[H]
    \centering
    \caption{SIR模型符号表}
    \label{table:SIR模型符号表}
    \begin{tabular}{ll}
        \hline
        符号     & 含义         \\
        \hline
        S        & 易感者       \\
        I        & 感染者       \\
        R        & 康复者       \\
        $\alpha$ & \PText{S}{I} \\
        $\beta$  & \PText{I}{R} \\
        \hline
    \end{tabular}
\end{table}
\par 感染机制如下:
\begin{align}
    S(t) & \xrightarrow \alpha I(t) \\
    I(t) & \xrightarrow \beta R(t)
\end{align}
\par 可以用积分方程表示为
\begin{align}
    \dt{S} & = -\alpha SI          \\
    \dt{I} & = \alpha SI - \beta I \\
    \dt{R} & = \beta I
\end{align}
\subsection{使用工具}
使用python实现爬虫获取数据(见附录\ref{appendix:数据})、
使用scipy库进行积分求解及数据拟合(见附录\ref{appendix:数据拟合结果})、
使用pyecharts库绘制图像。