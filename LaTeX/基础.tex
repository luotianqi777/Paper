\section{研究基础}
\subsection{研究概况}
疫情发生以来,
众多国内外不同领域的专家学者对此高度关注,
从各个层面深度剖析,
运用各种方法预测疫情趋势。
其中以基于$SIR$模型的$SEIR$模型最为多见,
众多专家学者依据COVID-19的特点通过$SEIR$模型进行拟合分析,
得出诸多研究成果。
\subsection{文献综述}
\par 早在18世纪中期,
就有人提出传染病的传播模型,
时至今日,
人们建立各种各样的传染病模型来预测疫情,
并将其应用到传染病之外的领域。
\par 传染病模型的首次提出为\citeauthor{对流行病数学理论的贡献}于\citeyear{对流行病数学理论的贡献}年
针对黑死病问题提出的$SIR$仓室模型\cite{对流行病数学理论的贡献},
$SIR$模型中将人群分为易感染者$S$(Susceptible)、感染者$I$(Infective)、移出者$R$(Removal)
(但在后来的模型定义中将原本的移出者$R$改为康复者$R$(Recovered)),
建立人群间转移的微分方程模型来模拟传染病的传播过程;
\citeauthor{Kermack-McKendrick确定性流行病模型的推广}于\citeyear{Kermack-McKendrick确定性流行病模型的推广}年
为经典的$SIR$模型做出推广\cite{Kermack-McKendrick确定性流行病模型的推广},
后来的$SIS$、$SEIR$、$SEIS$、$MSEIR$等众多模型都是以$SIR$模型为基础衍生出来的。
\par $SEIR$模型广泛用于传染病预测:
王小莉等人于\citeyear{应用SEIR模型预测2009年甲型H1N1流感流行趋势}年
根据不同因素下(年龄、地区、生活习惯等)的数据建立$SEIR$模型,
探究预测甲型H1N1流感在不同因素下的传播情况\cite{应用SEIR模型预测2009年甲型H1N1流感流行趋势};
王晓红于\citeyear{一类具有潜伏期的SEIR手足口病模型的研究}年
将$SEIR$模型用于预测手足口病的疫情趋势\cite{一类具有潜伏期的SEIR手足口病模型的研究},
并考虑了康复者复发的情况(即后来的$SEIRS$模型),
得到了足口病的基本再生数并对模型进行了动力学分析:平衡点的存在性和稳定性。
\par 不仅在传染病预测领域广受欢迎,
众多学者在其他领域用$SEIR$模型做出了许多成果:
杨旭颖等人于\citeyear{基于SEIR的社交网络信息传播模型}年
将$SEIR$模型用于模拟社交网络的信息传播机制\cite{基于SEIR的社交网络信息传播模型},
目的是预测社交网络上的信息传播趋势;
崔金栋等人于\citeyear{基于改良SEIR模型的微博话题式信息传播研究}年
以微博信息传播中的SEIR模型为出发点,
综合考虑微博网络中话题式信息的衍生特性,
构建改良式的微博话题式信息传播H-SEIR模型\cite{基于改良SEIR模型的微博话题式信息传播研究},
通过MATLAB进行模拟仿真,
对微博中话题式信息传播影响因素和对应的控制策略进行研究。
\par $SEIR$模型广泛应用于本次COVID-19疫情预测:
赵世磊等人通过$SUQC$模型\cite{通过流行病学建模表征传播和确定COVID-19的控制策略}
(在$SEIR$模型基础上加入潜伏期感染和隔离条件)
基于官方每日发布的数据不断调整隔离参数及传染率,
最终得出疫情在不同阶段的隔离率,
确诊率及病毒生殖率(即传播率)并对疫情最终影响人数做出预测,
其参数分段考虑的方式值得借鉴;
周慧娟等人通过$SEIQ$模型\cite{中国COVID-19爆发的流行动力学模型和控制}
(在$SEIR$模型基础上加入隔离人群$Q$)
通过$Montecarlo$方法模拟了两个独立的泊松过程
(分别是每日暴露病例和个体潜伏时间)
讨论在不同环境的不同隔离措施下病毒的感染率;
金华潘等人于\citeyear{2019年冠状病毒疾病控制策略的有效性:SEIR动态建模研究}年通过对$SEIR$模型做出一系列改动\cite{2019年冠状病毒疾病控制策略的有效性:SEIR动态建模研究}
(其中包括超级传染,治疗等)计算了2月16日之前的病毒基础生殖率,
该模型虽未考虑隔离情况,
但其对多种人群的引入值得借鉴;
战军湛等人基于$SEIR$模型借助人口迁移学通过城市间人口流动数据
\cite{结合人类迁移数据的2019年冠状病毒疾病在中国传播的建模与预测}预测疫情在城市间的传播,
但疫情期间人口流动严重停滞导致城市间传播效率低,
其以城市为基本单元的思考方式值得学习;
彭鹏等人基于广义动力学模型\cite{基于动力学模型的中国COVID-19流行病学分析}
引入七个状态(在原有的四个状态基础上新加入隔离、死亡、免疫)结合官方发布数据,
预测疫情进展并对疫情持续时间做出估计。
张良禄等人在\citeyear{SEIR建模分析与新型冠状病毒的斗争何时在武汉结束}年通过基础$SEIR$建模\cite{SEIR建模分析与新型冠状病毒的斗争何时在武汉结束},
将1月22日到2月12日间的数据以2月7日为界限分别建立了模型,
得出一系列基础参数并给出了预测结果,
但其对疫情过于乐观,
并未考虑隔离及潜伏期可传染,
不适用于长期预测。
% \citeauthor{估计2020年公主邮轮船上2019年新型冠状病毒的无症状比率}在\citeyear{估计2020年公主邮轮船上2019年新型冠状病毒的无症状比率}年使用哈密顿蒙特卡洛($HMC$)的贝叶斯框架\cite{估计2020年公主邮轮船上2019年新型冠状病毒的无症状比率}
% 预测封闭的小范围环境(游轮环境)的未感染人数,
% 可以用来粗略估计封闭的大范围环境(封闭下的城市)的未感染人数;
\par 并有学者使用不同的方法对COVID-19的参数做出了估计,其结果可用于本文结论参考:
李静华等人运用五种不同的方法\cite{估计2019年新型冠状病毒的流行性:数学建模研究}
(分别是指数增长、最大似然法、顺序贝叶斯方法、时间序列分析、$SEIR$模型)
对病毒在2月8日前后及整个疫情期间的传播率做出了估计;
\citeauthor{协调基本生殖数量及其不确定性的早期暴发估计:新型冠状病毒(SARS-CoV-2)暴发的框架和应用}
% 认为单纯的将病毒生殖数看作一个常量已不适I用于COVID-19,故
将病毒基本生殖数划分为三个关键量\cite{协调基本生殖数量及其不确定性的早期暴发估计:新型冠状病毒(SARS-CoV-2)暴发的框架和应用}
(分别为指数增长率,平均生成间隔,生成间隔离散度)并结合官方数据来估算病毒的传染率;
% 其思想值得借鉴,
% 将一些参数(如:传染率,治愈率)设计为一个与时间有关的分段函数
% (以隔离措施及医学进展的时间点为断点)比较合适。
上述两篇文献均采用与本文不同的拟合方式,
其结论有一定的参考价值。

\subsection{理论基础}
\par 在介绍$SEIR$模型前,
需要先说明$SIR$模型,
$SEIR$模型是由$SIR$模型发展而来的。
\par \citeauthor{对流行病数学理论的贡献}在\citeyear{对流行病数学理论的贡献}年研究黑死病时提出了仓室模型,模型中将人口分为三类:
\begin{itemize}
    \item 易感者(susceptibles),$S$人群
    \item 感染者(infectives),$I$人群
    \item 康复者(recovered),$R$人群
\end{itemize}
\par 将其称为$SIR$
\cite{对流行病数学理论的贡献}模型。
\citeauthor{Kermack-McKendrick确定性流行病模型的推广}在\citeyear{Kermack-McKendrick确定性流行病模型的推广}年对其进行了推广\cite{Kermack-McKendrick确定性流行病模型的推广},证明其广泛的适用性。
\par $SIR$模型的建立基于几个假设\cite{对流行病数学理论的贡献}:
\begin{itemize}
    \item 人口总数保持常量(包含预测死亡)
    \item 单位时间$t$传染人数与$S$和$I$人数成正比,即$S\to I = \alpha SI$
    \item 单位时间$t$康复人数与$I$成正比,即$I\to R = \beta I$
\end{itemize}
\begin{table}[H]
    \centering
    \caption{SIR模型符号表}
    \label{table:SIR模型符号表}
    \begin{tabular}{ll}
        \hline
        符号     & 含义         \\
        \hline
        S        & 易感者       \\
        I        & 感染者       \\
        R        & 康复者       \\
        $\alpha$ & \PText{S}{I} \\
        $\beta$  & \PText{I}{R} \\
        \hline
    \end{tabular}
\end{table}
\par 感染机制如下:
\begin{align}
    S(t) & \xrightarrow \alpha I(t) \\
    I(t) & \xrightarrow \beta R(t)
\end{align}
\par 可以用积分方程表示为
\begin{align}
    \dt{S} & = -\alpha SI          \label{math:SIR_S}  \\
    \dt{I} & = \alpha SI - \beta I  \label{math:SIR_I} \\
    \dt{R} & = \beta I\label{math:SIR_R}
\end{align}
\par 对式(\ref{math:SIR_S})的说明:
\par 易感者转变为感染者的人数为$\text{易感者}\times\text{接触率}\times\text{感染率}$,
其中接触率为$\frac{\text{感染者}}{\text{总人数}}$。
\par 总人数是一个常数,
规定$\alpha=\frac{\text{感染率}}{\text{总人数}}$,
得出$\dt{S}$:
\begin{align*}
    \dt{S} & = -\text{易感者}\times\text{接触率}\times\text{感染率} \\
           & = -S\cdot\frac{I}{\text{总人数}}\cdot\text{感染率}     \\
           & = -S\cdot I\cdot\frac{\text{感染率}}{\text{总人数}}    \\
           & = -S\cdot I \cdot \alpha
\end{align*}
\par 对式(\ref{math:SIR_R})的说明:
\begin{align*}
    \dt{R} & = \text{感染者}\times\text{治愈率} \\
           & =I\cdot \beta
\end{align*}
\par 对式(\ref{math:SIR_I})的说明:
\begin{align*}
    \dt{I} & = -\dt{S} - \dt{R}                     \\
           & = S\cdot I \cdot \alpha - I\cdot \beta
\end{align*}
\par $SEIR$模型是在$SIR$模型基础上加入病毒携带者$E$(Exposed)得来的,
详细模型将在\ref{sec:SEIR}进行说明。
\subsection{使用工具}
本文使用python实现爬虫动态获取所需数据、
使用scipy库进行高效率的积分求解及数据拟合、
使用pyecharts库绘制较为美观的图像。