% 模型描述
\section{模型}
\subsection{传统SIR模型}
\citeauthor{对流行病数学理论的贡献}在\citeyear{对流行病数学理论的贡献}年研究黑死病时提出了仓室模型,模型中将人口分为三类:
\begin{itemize}
    \item 易感者(susceptibles),$S$人群
    \item 感染者(infectives),$I$人群
    \item 康复者(recovered),$R$人群
\end{itemize}
\par 根据其人群划分将其称为$SIR$
\cite{对流行病数学理论的贡献}模型。
\citeauthor{Kermack-McKendrick确定性流行病模型的推广}在\citeyear{Kermack-McKendrick确定性流行病模型的推广}年对其进行了推广\cite{Kermack-McKendrick确定性流行病模型的推广},证明其广泛的适用性。
\par $SIR$模型的建立基于几个假设\cite{对流行病数学理论的贡献}:
\begin{itemize}
    \item 人口总数保持常量(包含死亡人数)
    \item 单位时间$t$传染人数与$S$和$I$人数成正比,即$S\to I = \alpha SI$
    \item 单位时间$t$康复人数与$I$成正比,即$I\to R = \beta I$
\end{itemize}
\subsubsection{符号表}
\begin{table}[H]
    \centering
    \caption{SIR模型符号表}
    \label{table:SIR模型符号表}
    \begin{tabular}{ll}
        \hline
        符号     & 含义                     \\
        \hline
        S        & 易感者                   \\
        I        & 感染者                   \\
        R        & 康复者                   \\
        $\alpha$ & $S$群体变为$I$群体的概率 \\
        $\beta$  & $I$群体变为$R$群体的概率 \\
        \hline
    \end{tabular}
\end{table}
\subsubsection{模型}
\par 感染机制如下:
\begin{align}
    S(t) & \xrightarrow \alpha I(t) \\
    I(t) & \xrightarrow \beta R(t)
\end{align}
可以用积分方程表示为
\begin{align}
    \dt{S} & = -\alpha SI          \\
    \dt{I} & = \alpha SI - \beta I \\
    \dt{R} & = \beta I
\end{align}
\subsection{对于模型表述方式的讨论}
\par 经典SIR模型的表述方式简洁直观,但在扩建模型时会产生一些问题:
由表\ref{table:SIR模型符号表}可预见,随着划分人群的增多,人群间感染机制也随之复杂,不同人群之间转变率的符号也会增多或者改变含义,这会使读者对符号的理解产生负面的路径依赖影响,对于模型的扩建是十分不利的。
\par 本文将采用另一种较为抽象的模型表述方式,以此来减轻读者的思维负担。
\subsubsection{新的符号表}
\begin{table}[H]
    \centering
    \caption{新的模型符号表}
    \begin{tabular}{ll}
        \hline
        符号       & 含义                              \\
        \hline
        S          & 易感者                            \\
        I          & 感染者                            \\
        R          & 康复者                            \\
        $\P{a}{b}$ & a群体变为b群体的概率              \\
        $\T{a}{b}$ & 单位时间$t$内a群体变为b群体的人数 \\
        \hline
    \end{tabular}
\end{table}
\par 将a到b群体的转换概率用$\P{a}{b}$表示,单位时间内a到b群体转变的人数为$\T{a}{b}$。
\subsubsection{新的模型表述}
\par 感染机制如下:
\begin{align}
    S(t)\xrightarrow{\P{S}{I}}I(t) \Rightarrow \TP{S}{I}{SI} \\
    I(t)\xrightarrow{\P{I}{R}}R(t) \Rightarrow \TP{I}{R}{I}
\end{align}
\par 可将其简化为:
\begin{align}
    \TP{S}{I}{SI} \\
    \TP{I}{R}{I}
\end{align}
\par 将所有人群放到一个集合$\mathbb{A}$中,则模型的积分方程为:
\begin{equation}
    \dt{a} = \sum\left(\T{b}{a}-\T{a}{b}\right)
\end{equation}
\par 其中$a\in\mathbb{A}$,$b\in\mathbb{A}$且$b\not=a$,不在感染机制中的$\T{a}{b}$为$0$。
\par 值得注意的是,该积分式对本文中提到的所有模型都适用,这意味着我们将不必再关心模型的积分式,只需关注模型新引入的群体及其引起的感染机制$\T{a}{b}$变化即可。
\par 现在,可以仅通过感染机制来方便清晰的描述$SIR$模型:
\begin{itemize}
    \item 人群
          \subitem $S$:易感者
          \subitem $I$:感染者
          \subitem $R$:康复者
    \item 感染机制
          \subitem
          \begin{align}
              \TP{S}{I}{SI} \\
              \TP{I}{R}{I}
          \end{align}
\end{itemize}
\subsection{SEIR模型}
考虑到易感人群接触到感染者后不会立即患病,而是经过一段时间潜伏期,即携带病毒还未患病,将该类人群定义为携带者人群,在COVID-19中这类人群会通过检测试剂等方式被被诊断为疑似病例。
\par
$SEIR$模型具有广泛的适用性,历来众多国内外学者通过其来研究传染病的传播
\cite{应用SEIR模型预测2009年甲型H1N1流感流行趋势,一类具有潜伏期的SEIR手足口病模型的研究},
或将其用于其它信息传播方向的研究
\cite{基于SEIR的社交网络信息传播模型,基于改良SEIR模型的微博话题式信息传播研究},
最近还有众多学者用其来预测本次COVID-19的传播趋势
\cite{通过流行病学建模表征传播和确定COVID-19的控制策略,中国COVID-19爆发的流行动力学模型和控制,2019年冠状病毒疾病控制策略的有效性:SEIR动态建模研究,结合人类迁移数据的2019年冠状病毒疾病在中国传播的建模与预测,基于动力学模型的中国COVID-19流行病学分析,SEIR建模分析与新型冠状病毒的斗争何时在武汉结束}。

\begin{itemize}
    \item 人群
          \subitem $S$:易感者
          \subitem $I$:感染者
          \subitem $R$:康复者
          \subitem $E$:携带者
    \item 感染机制
          \subitem
          \begin{align}
              \TP{S}{E}{S(I+E)} \\
              \TP{E}{I}{E}      \\
              \TP{I}{R}{I}
          \end{align}
\end{itemize}
\subsection{SEIRD模型}
在$SEIR$的基础上加入死亡人群,使其与易感者人群分离开。
\begin{itemize}
    \item 人群
          \subitem $S$:易感者
          \subitem $I$:感染者
          \subitem $R$:康复者
          \subitem $E$:携带者
          \subitem $D$:病逝者
    \item 感染机制
          \subitem
          \begin{align}
              \TP{S}{E}{S(I+E)} \\
              \TP{E}{I}{E}      \\
              \TP{I}{R}{I}      \\
              \TP{I}{D}{I}
          \end{align}
\end{itemize}
\subsection{SEIRS模型}
考虑到治愈者有复发的可能,康复者有一定比例转变为感染者。
\begin{itemize}
    \item 人群
          \subitem $S$:易感者
          \subitem $I$:感染者
          \subitem $R$:康复者
          \subitem $E$:携带者
          \subitem $D$:病逝者
    \item  感染机制
          \subitem
          \begin{align}
              \TP{S}{E}{S(I+E)} \\
              \TP{E}{I}{E}      \\
              \TP{I}{R}{I}      \\
              \TP{I}{D}{I}      \\
              \TP{R}{I}{R}
          \end{align}
\end{itemize}